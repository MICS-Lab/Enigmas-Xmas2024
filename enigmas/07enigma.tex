\documentclass[a4paper, top=10mm]{article}
%for writing from the top
\usepackage{fullpage}
%for math
\usepackage{amsmath}
\usepackage{mathrsfs}
\usepackage{amsthm}
%for images
\usepackage{graphicx}
%for color
\usepackage{xcolor}
%for title
\title{\textbf{\huge{Santa's Gift Distribution Order}}}
\author{Enigma n\textsuperscript{o}6}
\date{19\textsuperscript{th} December 2024}

\newtheorem*{hint}{Hint}

\addtolength{\voffset}{-2cm}
\addtolength{\textheight}{5cm}


\begin{document}
	\maketitle
	
	Santa is planning the perfect gift distribution for a family gathering. He has 10 gifts for kids and 10 gifts for adults, and he needs to arrange the order in which the gifts are handed out.
	
	However, there’s one important rule: at no point during the distribution should the number of adult gifts given out exceed the number of kids’ gifts.
	
	For example, a valid distribution order could be: Kids, Kids, Adults, Kids, Adults, Adults..., while an invalid one might start as: Adults, Adults, Kids....
	
	How many different valid ways can Santa arrange the distribution of the 20 gifts under this rule?
	
	\begin{center}
		\includegraphics[width=0.8\linewidth]{07presents.png}
	\end{center}
	
	% Answer: 16796
	% (catalan number C_10)
	% https://oeis.org/A000108
	% https://en.wikipedia.org/wiki/Catalan_number#Applications_in_combinatorics
	
\end{document}